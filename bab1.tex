%!TEX root = ./template-skripsi.tex
%-------------------------------------------------------------------------------
% 								BAB I
% 							LATAR BELAKANG
%-------------------------------------------------------------------------------

\chapter{PENDAHULUAN}

\section{Latar Belakang}
Kerja praktik adalah salah satu aktivitas bagi mahasiswa untuk mengaplikasikan ilmu teoretis yang didapat selama perkuliahan. Dengan adanya kegiatan kerja praktik, diharapkan dapat mempersiapkan untuk menghadapi dunia kerja. Kegiatan kerja praktik ini bersifat wajib untuk seluruh mahasiswa Departemen Teknik Elektro dan Teknologi Informasi.

Sejak pandemi Covid-19, perkembangan IoT menjadi sangat cepat untuk memenuhi kebutuhan masyakat yang menjadi serba daring. Hampir semua perangkat elektronik telah terhubung secara nirkabel dengan adanya IoT tak terkecuali dalam bidang pertanian. Penerapan IoT dalam bidang pertanian, yaitu suatu sistem pertanian dengan sumber daya dan pengendaliannya yang lebih efisien.

Indonesia adalah negara agraris dengan lahan pertanian yang luas. Iklim tropis di Indonesia juga mendukung banyaknya variasi tanaman, sehingga pengembangan di bidang pertanian berbasis IoT diharapkan dapat meningkatkan produktivitas pertanian di Indonesia. Sistem pertanian yang terotomasi dan terhubung ke internet biasa dikenal sebagai smart farming.

Tidak hanya di Indonesia, di negara maju teknologi smart farming sedang berkembang dengan pesat. Dengan adanya teknologi smart farming maka diharapkan agar industri pertanian di Indonesia dapat berkompetisi di pasar global. Peningkatan produktivitas yang didapat dari teknologi ini juga dapat berguna untuk meningkatkan bala bantuan pangan untuk orang yang membutuhkannya.

\section{Tujuan}
Habeo perfecto in sea. Ea deleniti gloriatur pri, paulo mediocrem incorrupte sea ei. Ad mollis scripta per. Incorrupte sadipscing ne mel. Mel ex nonumy malorum epicurei. Ne per tota mollis suscipit. Ullum labitur vim ut, ea dicit eleifend dissentias sit. Duis praesent expetenda ne sed. Sit et labitur albucius elaboraret. Ceteros efficiantur mei ad. Hendrerit vulputate democritum est at, quem veniam ne has, mea te malis ignota volumus.


\section{Waktu dan Tempat Pelaksanaan}
Kerja praktik ini dilaksanakan dinsalah satu perusahaan swasta yang bergerak di bidang IoT pada 4 Oktober 2021 – 4 Januari 2021. Rincian tempat pelaksanaan kerja praktik ini adalah sebagai berikut.

\noindent
\begin{tabular}{@{}p{0.13\textwidth}@{: }p{0.85\textwidth}}
  Tempat & PT. Inamas Sintesis Teknologi \\ 
  Alamat & Representative Office, Jalan Bunga Pikgondang, Condongcatur, Kec. Depok, Kabupaten Sleman, Daerah Istimewa Yogyakarta 552781 \\
  Sub divisi & Python Developer
\end{tabular}

PT. Inamas Sintesis Teknologi, sebagai tempat kerja praktik memberikan pendampingan untuk memastikan kerja praktik dapat berjalan dengan lancar. Pembimbing kerja praktik ini adalah sebagai berikut.

\noindent
\begin{tabular}{@{}p{0.2\textwidth}@{: }p{0.85\textwidth}}
  Pembimbing I & Meizar Raka Rimadana \\
  Pembimbing II	& Alwin Ihza Farandi
\end{tabular}

Pelaksanaan kerja praktik ini dilakukan secara bauran, daring dan luring. Hal tersebut bertujuan untuk mencegah penularan Covid-19. Mekanisme dan metode kerja praktik secara daring menggunakan media komunikasi WhatsApp, sedangkan secara luring pertemuan tiga kali setiap hari Senin, Selasa, dan Jumat.


\section{Batasan Masalah}
Eros reprimique vim no. Alii legendos volutpat in sed, sit enim nemore labores no. No odio decore causae has. Vim te falli libris neglegentur, eam in tempor delectus dignissim, nam hinc dictas an.


\section{Metode Pengumpulan Data}
Pro omnium incorrupte ea. Elitr eirmod ei qui, ex partem causae disputationi nec. Amet dicant no vis, eum modo omnes quaeque ad, antiopam evertitur reprehendunt pro ut. Nulla inermis est ne. Choro insolens mel ne, eos labitur nusquam eu, nec deserunt reformidans ut. His etiam copiosae principes te, sit brute atqui definiebas id.

Et affert civibus has. Has ne facer accumsan argumentum, apeirian hendrerit persequeris pro ex. Suscipit vivendum sensibus mea at, vim ei hinc numquam, at dicit timeam dissentiet mel. At patrioque intellegebat sea, error argumentum dissentias sea in.


\section{Sistematika Penulisan}
\noindent
\textbf{BAB I}

Berisi latar belakang, tujuan, manfaat, batasan masalah, waktu dan tempat pelaksanaan, dan metodologi penulisan laporan kerja praktik.\\

\noindent
\textbf{BAB II}

Berisi profil singkat, produk, struktur organisasi, dan visi dan misi PT. Inamas Sintesis Teknologi.\\

\noindent
\textbf{BAB III}

Berisi teori dan penjelasan umum teknologi dan metode yang digunakan selama pelaksanaan kerja praktik.\\

\noindent
\textbf{BAB IV : HASIL DAN PEMBAHASAN}

Pada bab ini dijelaskan hasil penelitian dan pembahasannya.\\

\noindent
\textbf{BAB V : KESIMPULAN DAN SARAN}

Berisi kesimpulan dari kerja praktik yang telah dilaksanakan serta saran untuk pengujian selanjutnya.\\

\noindent
\textbf{LAMPIRAN}

Berisi kesimpulan dari kerja praktik yang telah dilaksanakan serta saran untuk pengujian selanjutnya.\\

% Baris ini digunakan untuk membantu dalam melakukan sitasi
% Karena diapit dengan comment, maka baris ini akan diabaikan
% oleh compiler LaTeX.
\begin{comment}
\bibliography{daftar-pustaka}
\end{comment}
