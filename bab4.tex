%!TEX root = ./template-skripsi.tex
%-------------------------------------------------------------------------------
%                            BAB IV
%               		HASIL DAN PEMBAHASAN
%-------------------------------------------------------------------------------

\chapter{PEMBAHASAN}

\section{\textit{Class} MQTT}
\textit{Class} MQTT digunakan untuk membuat objek MQTT. Objek MQTT memiliki empat argumen, yaitu \textit{address} sebagai \textit{broker} MQTT, porta yang digunakan oleh koneksi MQTT, mode untuk menentukan apakah objek MQTT yang dibuat berperan sebagai subscriber atau publisher.

Objek MQTT memiliki empat buah \textit{method}, yaitu on\_connect yang akan dijalankan ketika objek MQTT telah berhasil terkoneksi dengan broker MQTT, on\_subscribe yang akan dijalankan ketika objek MQTT diatur sebagai subscriber dan berhasil berlangganan ke topik yang telah ditentukan, on\_publish yang akan dijalankan ketika objek MQTT diatur sebagai publisher dan akan mencetak pesan yang berhasil dikirimkan ke satu topik pada konsol, dan publish untuk mengirimkan pesan ke topik yang telah ditentukan.

\section{\textit{Class} DatabaseWrapper}
Habeo perfecto in sea. Ea deleniti gloriatur pri, paulo mediocrem incorrupte sea ei. Ad mollis scripta per. Incorrupte sadipscing ne mel. Mel ex nonumy malorum epicurei.

\section{\textit{Class} Growlight}
Habeo perfecto in sea. Ea deleniti gloriatur pri, paulo mediocrem incorrupte sea ei. Ad mollis scripta per. Incorrupte sadipscing ne mel. Mel ex nonumy malorum epicurei.

\section{\textit{Class} Arduino}
Habeo perfecto in sea. Ea deleniti gloriatur pri, paulo mediocrem incorrupte sea ei. Ad mollis scripta per. Incorrupte sadipscing ne mel. Mel ex nonumy malorum epicurei.
			
% Baris ini digunakan untuk membantu dalam melakukan sitasi.
% Karena diapit dengan comment, maka baris ini akan diabaikan
% oleh compiler LaTeX.
\begin{comment}
\bibliography{daftar-pustaka}
\end{comment}