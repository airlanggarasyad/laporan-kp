%!TEX root = ./template-skripsi.tex
%-------------------------------------------------------------------------------
%                            BAB III
%               		METODOLOGI PENELITIAN
%-------------------------------------------------------------------------------

\chapter{DASAR TEORI}

\section{\textit{Internet of Things}}
\textit{Internet of Things} atau IoT adalah suatu konsep ketika suatu perangkat dapat dibaca, dikenal, dilacak, dan diakses dengan menggunakan internet. Perangkat yang dapat terkoneksi dengan internet tidak hanya sebatas perangkat elektronik, tetapi perangkat seperti kendaraan dan beberapa peralatan lainnya seperti hewan dan tanaman. IoT juga dapat didefinisikan sebagai suatu infrastruktur global yang memungkinkan layananyang saling terhubung berdasarkan teknologi informasi dan komunikasi yang bersifat \textit{interoperable}. 

IoT adalah suatu paradigma yang mempertimbangkan kehadiran berbagai objek atau perangkat yang dapat berkomunikasi satu sama lain melalui komunikasi baik secara nirkabel dan kabel. Objek fisik atau maya teresebut harus dapat berkomunikasi, dilacak, dan berinteraksi dengan lingkungannya. Objek fisik meliputi barang-barang seperti komputer dan ponsel pintar.


\section{Pemrograman Berorientasi Objek}
Pemrograman berorientasi objek adalah salah satu paradigma pemrograman yang banyak digunakan oleh pengembang di dunia. Hampir seluruh aplikasi yang terdapat di pasaran saat ini dikembangkan dengan paradigma pemrograman berorientasi objek. Pemrograman berorientasi objek berfokus pada model objek itu sendiri, sedangkan pemrograman prosedural berfokus pada algoritmanya. Terdapat empat konsep dasar dalam pemrograman berbasis objek, yaitu \textit{abstraction}, \textit{encapsulation}, dan \textit{inheritances}.

	\subsection{\textit{Abstraction}}
	\textit{Abstraction} adalah suatu mekanisme untuk menyembunyikan proses yang terjadi pada suatu objek. Objek tersebut hanya akan menyediakan apa yang dibutuhkan saat ini. Pada saat melakukan \textit{abstraction}, pengembang hanya perlu fokus pada objek tersebut tanpa harus tahu proses apa yang terjadi.

	\subsection{\textit{Encapsulation}}
	\textit{Encapsulation} adalah membungkus suatu kelas, baik method atau atributnya agar tidak diakses oleh kelas lainnya. Pada \textit{encapsulation} dikenal tiga buah hak akses modifier yang terdiri dari \textit{private}, \textit{public}, dan \textit{protected}. \textit{Private} hanya memberikan akses pada kelas itu sendiri. \textit{Public} memberikan hak akses method dan atribut agar dapat diakses oleh seluruh kelas. \textit{Protected} hanya memberikan akses ke kelas itu sendiri dan turunannya.

	\subsection{\textit{Inheritance}}
	\textit{Inheritance} atau pewarisan adalah suatu konsep ketika suatu objek mendapat sifat dan perilaku dari objek lainnya. Jika ingin membuat suatu kelas yang fungsinya sudah ada di kelas sebelumnya maka pengembang cukup mewariskan kelas tersebut dan tidak perlu menulis ulang kodenya.
	
\section{Bahasa Pemrograman Python}
Python adalah salah satu bahasa pemrograman paling populer yang dikembangkan oleh Guido van Rossum pada dekade 90-an. Popularitas yang tinggi diakibatkan oleh kode yang mudah untuk dibaca dan \textit{syntax} yang tidak serumit bahasa pemrograman lain seperti C dan C++. Bahasa pemrograman Python termasuk sebagai bahasa pemrograman yang bersifat \textit{multiparadigm} atau mendukung lebih dari satu paradigma pemrograman, yaitu imperatif, fungsional, prosedural, dan berorientasi objek.

Karena sifatnya \textit{multiparadigm}, Python mendukung pemrograman berorientasi objek. Hampir seluruh program yang ditulis dengan Python dalam bentuk kelas dan objek. Python juga mendukung banyak basis data seperti MySQL, MongoDB, PostgreSQL, dan masih banyak lagi.


\section{\textit{Message Queuing Temetry Transport}}
Pada perangkat IoT dibutuhkan suatu protokol komunikasi yang ringan untuk transmisi data. Salah satu protokol komunikasi yang paling banyak digunakan adalah MQTT. Pada TCP/IP, MQTT berada di lapisan paling atas atau lapisan aplikasi. 

MQTT terdiri dari dua buah agen, yaitu \textit{broker} dan \textit{client}. \textit{Broker} berperan sebagai \textit{server}, sedangkan \textit{client} berperan sebagai perangkat yang saling bertukar pesan. \textit{Client} pada MQTT sendiri dibagi menjadi dua lagi, yaitu \textit{publisher} dan \textit{subscriber}. \textit{Publisher} adalah perangkat yang mengirimkan pesan ke satu topik dan \textit{subscriber} adalah perangkat yang berlangganan pada suatu topik dan akan menerima pesan yang dikirimkan ke topik tersebut.

Menurut studi yang dilakukan oleh Vergara et.al menunjukan bahwa protokol MQTT yang dijalankan pada sebuah perangkat Android memiliki konsumsi energi yang lebih sedikit jika dibandingkan dengan HTTP. Dari kelebihan-kelebihan tersebut terlihat bahwa MQTT cocok untuk digunakan pada aplikasi berbasis \textit{mobile} maupun \textit{web}.

Terdapat empat belas tipe kendali isyarat pada MQTT, yaitu CONNECT, CONNACK, PUBLISH, PUBACK, PUBREC, PUBREL, PUBCOMP, SUBSCRIBE, SUBACK, UNSUBSCRIBE, UNSUBACK, PINGERQ, PINGRESP, dan DISCONNECT.


\section{Komunikasi Serial}
Metode komunikasi yang banyak digunakan pada mikroprosesor atau mikrokontroler adalah komunikasi serial. Pada komunikasi serial, satu bit akan mengikuti bit lainnya, sehingga pada komunikasi serial hanya dibutuhkan satu buah kanal komunikasi. Karena hanya membutuhkan satu buah kanal komunikasi maka komunikasi serial dapat menghemat sumber daya sebanyak n kali dari komunikasi paralel. Komunikasi serial dapat dilakukan dengan tiga cara, yaitu \textit{asynchronous}, \textit{synchronous}, dan \textit{isochronous}.
	
\section{MongoDB}
MongoDB adalah salah satu basis data dengan konsep NoSQL yang berbasis dokumen. NoSQL adalah basis data yang bersifat tanpa relasi atau dapat mengelola basis data tanpa menggunakan \textit{query} yang kompleks. 

MongoDB memiliki performa yang lebih cepat dibandingkan dengan jenis SQL karena basis data ini menggunakan dokumen dengan format BSON (\textit{Binary JSON}). Selain itu MongoDB juga lebih mudah dalam pengelolaannya karena tidak memerlukan struktur tabel yang rumit. MongoDB juga mampu untuk menampung lebih banyak data yang kompleks karena menggunakan \textit{table scheme} yang dinamis. Kemampuan skalabilitas pada MongoDB juga lebih mudah jika dibandingkan dengan basis data SQL. MongoDB bersifat \textit{open-source} dan gratis.

	
% Baris ini digunakan untuk membantu dalam melakukan sitasi
% Karena diapit dengan comment, maka baris ini akan diabaikan
% oleh compiler LaTeX.
\begin{comment}
\bibliography{daftar-pustaka}
\end{comment}
